\section{Physics example: $H \rightarrow ZZ \rightarrow 4l$}
\justifying
\paragraph{}
The goal of this exercise is to reconstruct the Standard Model (SM) Higgs boson mass, using a selection targeting the four-lepton final state. This is considered a \textit{golden} channel to rediscovered the Higgs because:
\begin{itemize}
	\item there is a \textbf{\underline{ large signal to background ratio}} -- it is easy to discriminate between the peak of the reconstructed four-lepton mass ($m_{4l}$) and the overall flat background shape; 
	\item we have excellent \textbf{\underline{ mass resolution}} -- thanks to the great resolution power of CMS, we have optimal shape reconstruction of $m_{4l}$;
	\item it is a \textbf{\underline{ resolved final state}} -- detection of the four leptons in the final state ensures good discrimination of signal and background.
\end{itemize}

\begin{figure}[t]
	\centering
	\includegraphics[width=\textwidth]{images/CMS-HIG-19-001_Figure_004-a.pdf}
	\Caption{Reconstructed four-lepton invariant mass $m_{4l}$ with full Run2 data}{The SM Higgs boson signal with $m_H = 125\,\text{GeV}$, denoted as $H(125)$, and the $ZZ$ backgrounds are normalized to the SM expectation. The $Z+X$ background is normalized to the estimation from data.
	Figure taken from ref.~\cite{h4l_analysis}.}
	\label{higgs_plot}
\end{figure}