\chapter{Basic Functionalities}\label{chap:basics}

This chapter illustrates how to employ the most basic features of \columnflow.
By the end of it, you should be able to perform a calibration, apply a selection, calculate an observable and finally also produce the corresponding distribution for multiple processes.
Please note that this chapter is merely meant to summarize the most important aspects of these features.
For a more in-depth discussion and presentation, please consider Ref.~\cite{cf_repo}.

\section{The mother of all: TaskArrayFunctions}\label{sec:taskarrayfunc}

Before getting started with the details of the implementation, we will cover the basic structure of the most relevant building blocks of \columnflow.
These objects are all derived from the so-called \CCSPStlye{TaskArrayFunction}, which defines hooks and interfaces to propagate information from your objects to the \columnflow tasks.

This class of objects can for example explicitly define some runtime dependencies with the following member variables:
\begin{description}
	\item[uses] is a set of column names that are to be retrieved from disk.
	You can also provide other \CCSPStlye{TaskArrayFunction}s here, in which case their \texttt{uses} set is appended to this one.
	\item[produces] is a set of columns that are to be written to disk.
	You can also provide other \CCSPStlye{TaskArrayFunction}s here, in which case their \texttt{produces} set is appended to this one.
	\item[sandbox] is a hook that is propagated to the actual Task instance that is run and calls your module.
	This defines the software environment in which your module needs to run, which allows for a granular definition of the required software and can minimize the overhead of your software packages.
	
\end{description}

For convenience, all  \CCSPStlye{TaskArrayFunction}s provide decorators to easily define new modules:

\begin{lstlisting}[language=python]
	# assuming you want to define the TaskArrayFunction example
	@example(
		# for example, request the Jet pt, all Electron information and everything another 
		uses = {
			"Jet.pt",  # request transverse momentum for all jets
			"Electron.*",  # request all information for electrons
			some_other_TaskArrayFunction  # propagate everything some_other_TaskArrayFunction needs to this example TaskArrayFunction
	    },
	    # define which outputs are to be written to disk
	    produces={
	    	"some_fancy_output"
	    },
	    # define in what kind of software environment this module should be run
	    sandbox="some_cool_sandbox"
	)
	def your_new_example_module(events: ak.Array, **kwargs):
	    # this is the main body of your module, do something here...
\end{lstlisting}

Additionally, \CCSPStlye{TaskArrayFunction}s provide a set of hooks, three of which are of special importance and are briefly introduced in the following:
\begin{description}
	\item[init] defines instructions that are to be done when this object is first initialised.
	\item[requires] adds object-specific requirements on top of the pre-existing Task-level requirements.
	This allows to explicitly define dependencies and can for example ensure that the output of another module is calculated before starting with the current task.
	\item[setup] is run before actually entering the main body of your module that performs e.g.\ calculations.
	This is for example useful to parse the output of the aforementioned requirements such that your object can also use it.
	
\end{description}

These hooks can be added to an existing  \CCSPStlye{TaskArrayFunction} instance with dedicated decorators like so:

\begin{lstlisting}[language=python]
	# assuming you have defined your_new_example_module from the example above
	
	# define your init function
	@your_new_example_module.init
	def some_init_func_name(self):
	    # do something when your_new_example_module is first initialized
	    
	# define some special requirements for your module
	@your_new_example_module.requires
	def some_func_name_for_requires(self):
	    # add some requirements
	
	# do something before entering the main body of your module
	@your_new_example_module.setup
	def some_setup_func_name(self, **kwargs):
	    # prepare your module so it runs smoothly
\end{lstlisting}

In the following, these concepts will be shown with concrete details.
