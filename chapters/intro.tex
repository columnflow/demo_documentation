\chapter{Introduction to ColumnFlow}

ColumnFlow is intended as a back-end for analyses in order to facilitate processing large amounts of data.
It is purely python-based and employs multiple packages that are well-received and {-maintained} in the HEP community.
At the time of writing these instructions, the team of developer's purely consists of data analysts at the CMS experiment.
Therefore, this exercise is structured accordingly.
Please note that ColumnFlow is in principle designed in an experiment-agnostic way, such that it can also be extended to other use cases.

Additionally, please note that this hands-on exercise is not meant to fully document all available functionalities.
The purpose of this exercise is to give an overview of the most fundamental aspects and concepts that are available at the time of writing.
For a more comprehensive overview, please visit the \href{https://columnflow.readthedocs.io/en/latest/}{official documentation}. % might want to put this as a proper reference
In case of any questions are comments, feel free to contact the maintainers for example via the \href{https://github.com/columnflow/columnflow}{git repository}.

\begin{figure}[p]
	\centering
	\includegraphics[scale=0.8]{images/CF_tasks.png}
	\caption{\justifying{ColumnFlow task graph hierarchy}}
	\label{fig:task_graph}
\end{figure}